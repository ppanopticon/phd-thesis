%%%%%%%%%%%%%%%%%%%%%%%%%%%%%%%%%%%%%%%%%%%%%
% Text symbols.
%%%%%%%%%%%%%%%%%%%%%%%%%%%%%%%%%%%%%%%%%%%%%
\def\YOURPRODUCT/{\textsf{YOURproduct}} % use YOURPRODUCT as \YOURPRODUCT/


%%%%%%%%%%%%%%%%%%%%%%%%%%%%%%%%%%%%%%%%%%%%%
% Symbols.
%%%%%%%%%%%%%%%%%%%%%%%%%%%%%%%%%%%%%%%%%%%%%

%% A group (the group is specified in the sort field)
\newcommand*{\Agroupname}{Basics}

\newglossaryentry{symnatural}{sort={AAsymnatural},name=\ensuremath{\mathbb{N}},symbol=\ensuremath{\mathbb{N}},type={symbols},description={Set of natural numbers.}}
\newcommand{\symnatural}{{\glssymbol{symnatural}}}

\newglossaryentry{symreal}{sort={ABsymreal},name=\ensuremath{\mathbb{R}},symbol=\ensuremath{\mathbb{R}},type={symbols},description={Set of real numbers.}}
\newcommand{\symreal}{{\glssymbol{symreal}}}

\newglossaryentry{symcomplex}{sort={ACsymcomplex},name=\ensuremath{\mathbb{C}},symbol=\ensuremath{\mathbb{C}},type={symbols},description={Set of all complex numbers.}}
\newcommand{\symcomplex}{{\glssymbol{symcomplex}}}

\newglossaryentry{symand}{sort={ADsymand},name=\ensuremath{\wedge},symbol=\ensuremath{\wedge},type={symbols},description={Logical conjunction.}}
\newcommand{\symand}{{\glssymbol{symand}}}

\newglossaryentry{symor}{sort={AEsymor},name=\ensuremath{\vee},symbol=\ensuremath{\vee},type={symbols},description={Logical, inclusive disjunction.}}
\newcommand{\symor}{{\glssymbol{symor}}}

\newglossaryentry{symnot}{sort={AFsymnot},name=\ensuremath{\neg},symbol=\ensuremath{\neg},type={symbols},description={Logical negation.}}
\newcommand{\symnot}{{\glssymbol{symnot}}}

%% B group (the group is specified in the sort field)
\newcommand*{\Bgroupname}{Relational Algebra}

% this is the definition of a symbol entry, note that glossaryentry is defined; for convenience using a newcommand, glssymbol is printed
% see also the use of \ensuremath
\newglossaryentry{relation}{sort={BArelation},name=\ensuremath{\mathcal{R}},symbol=\ensuremath{\mathcal{R}},type={symbols},description={A relation. Sometimes with subscript, e.g., \( \relation_{\mathtt{name}} \) to specify name or index.}}
\newcommand{\relation}{\glssymbol{relation}}

\newglossaryentry{rankedrel}{sort={BBrankedrelation},name=\ensuremath{\relation^{O}},symbol=\ensuremath{\relation^{O}},type={symbols},description={A ranked, relation that exhibits a partial ordering of elements induced by \( O \). Sometimes used with subscript, e.g., \( \rankedrel_{\mathtt{name}} \) to specify name or index.}}
\newcommand{\rankedrel}{\glssymbol{rankedrel}}

\newglossaryentry{domain}{sort={BCdomain},name=\ensuremath{\mathcal{D}},symbol=\ensuremath{\mathcal{D}},type={symbols},description={A data domain of a relation. Sometimes with subscript, e.g., \(\domain_{\mathtt{name}}\), to specify name or index.}}
\newcommand{\domain}{\glssymbol{domain}}

\newglossaryentry{attribute}{sort={BDattribute},name=\ensuremath{\mathcal{A}},symbol=\ensuremath{\mathcal{A}},type={symbols},description={An attribute of a relation, i.e., a tuple \((\mathtt{name}, \domain)\). Sometimes with subscript to specify name or index.}}
\newcommand{\attribute}{\glssymbol{attribute}}

\newglossaryentry{attributep}{sort={BEattributep},name=\ensuremath{\mathcal{A}^{*}},symbol=\ensuremath{\mathcal{A}^{*}},type={symbols},description={An attribute that acts as a primary key.}}
\newcommand{\attributep}{\glssymbol{attributep}}

\newglossaryentry{attributef}{sort={BFfattributef},name=\ensuremath{\overline{\mathcal{A}}},symbol=\ensuremath{\overline{\mathcal{A}}},type={symbols},description={An attribute that acts as a foreign key.}}
\newcommand{\attributef}{\glssymbol{attributef}}

\newglossaryentry{tuple}{sort={BGtuple},name=\ensuremath{t},symbol=\ensuremath{t},type={symbols},description={A tuple \( \tuple \in \relation \) of attribute values \( a_i \in \domain_i \). Sometimes with subscript to specify the index of the tuple in \(  \relation \).}}
\newcommand{\tuple}{\glssymbol{tuple}}

\newglossaryentry{domainset}{sort={BHdomainset},name=\ensuremath{\mathbb{D}},symbol=\ensuremath{\mathbb{D}},type={symbols},description={The set system of all data domains \( \domain \) supported by a \acrshort{dbms}.}}
\newcommand{\domainset}{\glssymbol{domainset}}

\newglossaryentry{schema}{sort={BIschema},name=\ensuremath{\mathtt{SCH}},symbol=\ensuremath{\mathtt{SCH}},type={symbols},description={The schema of a relation \( \relation \), i.e., all attributes \(\schema (\relation) = (\attribute_1, \attribute_2, \ldots, \attribute_N)\) that make up \(\relation \).}}
\newcommand{\schema}{\glssymbol{schema}}

\newglossaryentry{selection}{sort={BJselection},name=\ensuremath{\sigma},symbol=\ensuremath{\sigma},type={symbols},description={The unary selection operator used to filter a relation \(\relation \). Usually printed with a subscript to describe the boolean predicate $S$, e.g. \( \selection_{S} \).}}
\newcommand{\selection}{\glssymbol{selection}}

\newglossaryentry{projection}{sort={BKprojection},name=\ensuremath{\pi},symbol=\ensuremath{\pi},type={symbols},description={The unary projection operator used to restrict the attributes \( \attribute \) of a relation \( \relation \). Usually used with subscript to list the projection attributes $P$, e.g., \( \projection_{\attribute_1,\attribute_2} \).}}
\newcommand{\projection}{\glssymbol{projection}}

\newglossaryentry{rename}{sort={BLrename},name=\ensuremath{\rho},symbol=\ensuremath{\rho},type={symbols},description={The unary rename operator used to rename the attributes \( \attribute \) of a relation \( \relation \). Usually used with subscript to list the rename operations, e.g., \( \rename_{\attribute_1 \rightarrow \attribute_2} \).}}
\newcommand{\rename}{\glssymbol{rename}}

\newglossaryentry{order}{sort={BMorder},name=\ensuremath{\omega},symbol=\ensuremath{\omega},type={symbols},description={The binary sort operator used to construct a ranked relation \( \rankedrel \). Usually printed with a subscript to list order attributes $O$, e.g., \( \order_{\attribute_1 \uparrow,\attribute_2 \downarrow} \).}}
\newcommand{\order}{\glssymbol{order}}

\newglossaryentry{limit}{sort={BNlimit},name=\ensuremath{\lambda},symbol=\ensuremath{\lambda},type={symbols},description={The unary k-selection operator restricts \( \relation \) to the first \( k \) tuples. Usually printed with a subscript to define \( k \), e.g., \( \limit_{k} \).}}
\newcommand{\limit}{\glssymbol{limit}}

\newglossaryentry{aggregation}{sort={BOlimit},name=\ensuremath{\gamma},symbol=\ensuremath{\gamma},type={symbols},description={The unary aggregation operator.}}
\newcommand{\aggregation}{\glssymbol{aggregation}}

\newglossaryentry{group}{sort={BPgroup},name=\ensuremath{\gamma},symbol=\ensuremath{\gamma},type={symbols},description={The unary group operator aggregates tuples in \( \relation \). Usually printed with a subscript to list the group attributes $G$, e.g., \( \group_{G} \).}}
\newcommand{\group}{\glssymbol{group}}

%% C group (the group is specified in the sort field)
\newcommand*{\Cgroupname}{Multimedia Retrieval}

\newglossaryentry{symmediacol}{sort={CAsymmediacol},name=\ensuremath{\mathbb{O}},symbol=\ensuremath{\mathbb{O}},type={symbols},description={The media collection domain \(\symmediacol = \lbrace o_1, o_2 , \ldots, o_N \rbrace\).}}
\newcommand{\symmediacol}{\glssymbol{symmediacol}}

\newglossaryentry{symfeatures}{sort={CBsymfeatures},name=\ensuremath{\mathbb{F}},symbol=\ensuremath{\mathbb{F}},type={symbols},description={The feature domain \(\symfeatures = \lbrace f_1, f_2, \ldots f_M \rbrace\).}}
\newcommand{\symfeatures}{\glssymbol{symfeatures}}

\newglossaryentry{symfeaturetransform}{sort={CCsymfeaturetransform},name=\ensuremath{\mathfrak{t}},symbol=\ensuremath{\mathfrak{t}},type={symbols},description={Feature transformation function \(\symfeaturetransform \colon \symmediacol \rightarrow \symfeatures \)}}
\newcommand{\symfeaturetransform}{\glssymbol{symfeaturetransform}}

\newglossaryentry{symdist}{sort={CDsymdist},name=\ensuremath{\mathfrak{d}},symbol=\ensuremath{\mathfrak{d}},type={symbols},description={Dissimilarity or distance function \(\symdist \colon \symfeatures \times \symfeatures \rightarrow \symreal\).}}
\newcommand{\symdist}{\glssymbol{symdist}}

\newglossaryentry{symsim}{sort={CEsymsim},name=\ensuremath{\mathfrak{s}},symbol=\ensuremath{\mathfrak{s}},type={symbols},description={Similarity function \(\symsim \colon \symfeatures \times \symfeatures \rightarrow [0, 1.]\).}}
\newcommand{\symsim}{\glssymbol{symsim}}

\newglossaryentry{symcorr}{sort={CFsymcorr},name=\ensuremath{\mathfrak{c}},symbol=\ensuremath{\mathfrak{c}},type={symbols},description={A correspondence function \(\symcorr \colon \symreal \rightarrow [0, 1]\).}}
\newcommand{\symcorr}{\glssymbol{symcorr}}


%% C group (the group is specified in the sort field)
\newcommand*{\Dgroupname}{System Model}

\newglossaryentry{symdfc}{sort={DAsymdfc},name=\ensuremath{\hat{\mathfrak{d}}},symbol=\ensuremath{\hat{\mathfrak{d}}},type={symbols},description={A \gls{dfc} \( \symdfc \colon \domain_q \times \domain_q \times \domain_{1} \cdot \times \domain_{n} \to \symnatural \)}}
\newcommand{\symdfc}{\glssymbol{symdfc}}

\newglossaryentry{symindex}{sort={DEsymindex},name=\ensuremath{\mathtt{INDEX}^{\relation}_{I}},symbol=\ensuremath{\mathtt{INDEX}^{\relation}_{I}},type={symbols},description={An index on relation \( \relation \) indexing attributes \( I \subset \schema(\relation) \).}}
\newcommand{\symindex}{\glssymbol{symindex}}

\newglossaryentry{symop}{sort={DFsymop},name=\ensuremath{\mathtt{OP}},symbol=\ensuremath{\mathtt{OP}},type={symbols},description={A (relational) database operation, typically acting on a relation \( \relation \).}}
\newcommand{\symop}{\glssymbol{symop}}

\newglossaryentry{symplan}{sort={DFsymplan},name=\ensuremath{\mathcal{P}},symbol=\ensuremath{\mathcal{P}},type={symbols},description={A query execution plan, i.e., a concatenation of $N$ operators \(\mathtt{OP}_1 \circ \mathtt{OP}_2, \ldots \circ \mathtt{OP}_N \).}}
\newcommand{\symplan}{\glssymbol{symplan}}

\newglossaryentry{symchangeop}{sort={DHsymchangeop},name=\ensuremath{\mathtt{OP}_c},symbol=\ensuremath{\mathtt{OP}_c},type={symbols},description={A change operation \( \mathtt{OP}_c(\relation, \cdot)\) that can either be an insert, update or delete.}}
\newcommand{\symchangeop}{\glssymbol{symchangeop}}

\newglossaryentry{symchange}{sort={DIsymchange},name=\ensuremath{\mathcal{C}},symbol=\ensuremath{\mathcal{C}},type={symbols},description={A change on relation \( \relation \) with \( \symchange = (\symchangeop, \relation, t, t^{\prime})\).}}
\newcommand{\symchange}{\glssymbol{symchange}}

%%%%%%%%%%%%%%%%%%%%%%%%%%%%%%%%%%%%%%%%%%%%%
% Math Symbols.
%%%%%%%%%%%%%%%%%%%%%%%%%%%%%%%%%%%%%%%%%%%%%
\DeclareMathOperator*{\argmin}{argmin}
\DeclareMathOperator*{\argmax}{argmax}

%%%%%%%%%%%%%%%%%%%%%%%%%%%%%%%%%%%%%%%%%%%%%
% Image.
%%%%%%%%%%%%%%%%%%%%%%%%%%%%%%%%%%%%%%%%%%%%%
\newcommand*{\pin}{\includegraphics[height=\heightof{M}]{figures/pin}}