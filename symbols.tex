%%%%%%%%%%%%%%%%%%%%%%%%%%%%%%%%%%%%%%%%%%%%%
% Text symbols.
%%%%%%%%%%%%%%%%%%%%%%%%%%%%%%%%%%%%%%%%%%%%%
\def\YOURPRODUCT/{\textsf{YOURproduct}} % use YOURPRODUCT as \YOURPRODUCT/


%%%%%%%%%%%%%%%%%%%%%%%%%%%%%%%%%%%%%%%%%%%%%
% Symbols.
%%%%%%%%%%%%%%%%%%%%%%%%%%%%%%%%%%%%%%%%%%%%%

%% A group (the group is specified in the sort field)
\newcommand*{\Agroupname}{Basics}

\newglossaryentry{symnatural}{sort={Asymnatural},name=\ensuremath{\mathbb{N}},symbol=\ensuremath{\mathbb{N}},type={symbols},description={Set of natural numbers.}}
\newcommand{\symnatural}{{\glssymbol{symnatural}}}

\newglossaryentry{symreal}{sort={Asymreal},name=\ensuremath{\mathbb{R}},symbol=\ensuremath{\mathbb{R}},type={symbols},description={Set of all real numbers.}}
\newcommand{\symreal}{{\glssymbol{symreal}}}

\newglossaryentry{symcomplex}{sort={Asymcomplex},name=\ensuremath{\mathbb{C}},symbol=\ensuremath{\mathbb{C}},type={symbols},description={Set of all complex numbers.}}
\newcommand{\symcomplex}{{\glssymbol{symcomplex}}}

%% B group (the group is specified in the sort field)
\newcommand*{\Bgroupname}{Relational Algebra}

% this is the definition of a symbol entry, note that glossaryentry is defined; for convenience using a newcommand, glssymbol is printed
% see also the use of \ensuremath
\newglossaryentry{symrel}{sort={Bsymrel},name=\ensuremath{\mathcal{R}},symbol=\ensuremath{\mathcal{R}},type={symbols},description={A relation. Sometimes used with subscript, e.g., \(\symrel_{\mathtt{name}}\) to specify name or index.}}
\newcommand{\symrel}{\glssymbol{symrel}}

\newglossaryentry{symdomain}{sort={Bsymdomain},name=\ensuremath{\mathcal{D}},symbol=\ensuremath{\mathcal{D}},type={symbols},description={A data domain of a relation. Sometimes used with subscript, e.g., \(\symdomain_{\mathtt{name}}\), to specify name or index.}}
\newcommand{\symdomain}{\glssymbol{symdomain}}

\newglossaryentry{symattr}{sort={Bsymattr},name=\ensuremath{\mathcal{A}},symbol=\ensuremath{\mathcal{A}},type={symbols},description={An attribute of a relation, i.e., a tuple \((\mathtt{name}, \symdomain)\). Always used with subscript to specify name.}}
\newcommand{\symattr}{\glssymbol{symattr}}

\newglossaryentry{symattrp}{sort={Bsymattrp},name=\ensuremath{\mathcal{A}^{*}},symbol=\ensuremath{\mathcal{A}^{*}},type={symbols},description={An attribute that acts as a primary key.}}
\newcommand{\symattrp}{\glssymbol{symattrp}}

%%%%%%%%%%%%%%%%%%%%%%%%%%%%%%%%%%%%%%%%%%%%%
% Math Symbols.
%%%%%%%%%%%%%%%%%%%%%%%%%%%%%%%%%%%%%%%%%%%%%
\DeclareMathOperator*{\argmin}{argmin}
\DeclareMathOperator*{\argmax}{argmax}

%%%%%%%%%%%%%%%%%%%%%%%%%%%%%%%%%%%%%%%%%%%%%
% Image.
%%%%%%%%%%%%%%%%%%%%%%%%%%%%%%%%%%%%%%%%%%%%%
\newcommand*{\pin}{\includegraphics[height=\heightof{M}]{figures/pin}}