%%%%%%%%%%%%%%%%%%%%%%%%%%%%%%%%%%%%%%%%%%%%%
% Text symbols.
%%%%%%%%%%%%%%%%%%%%%%%%%%%%%%%%%%%%%%%%%%%%%
\def\YOURPRODUCT/{\textsf{YOURproduct}} % use YOURPRODUCT as \YOURPRODUCT/


%%%%%%%%%%%%%%%%%%%%%%%%%%%%%%%%%%%%%%%%%%%%%
% Symbols.
%%%%%%%%%%%%%%%%%%%%%%%%%%%%%%%%%%%%%%%%%%%%%

%% A group (the group is specified in the sort field)
\newcommand*{\Agroupname}{Basics}

\newglossaryentry{symnatural}{sort={AAsymnatural},name=\ensuremath{\mathbb{N}},symbol=\ensuremath{\mathbb{N}},type={symbols},description={Set of natural numbers.}}
\newcommand{\symnatural}{{\glssymbol{symnatural}}}

\newglossaryentry{symreal}{sort={ABsymreal},name=\ensuremath{\mathbb{R}},symbol=\ensuremath{\mathbb{R}},type={symbols},description={Set of real numbers.}}
\newcommand{\symreal}{{\glssymbol{symreal}}}

\newglossaryentry{symcomplex}{sort={ACsymcomplex},name=\ensuremath{\mathbb{C}},symbol=\ensuremath{\mathbb{C}},type={symbols},description={Set of all complex numbers.}}
\newcommand{\symcomplex}{{\glssymbol{symcomplex}}}

\newglossaryentry{symand}{sort={ADsymand},name=\ensuremath{\wedge},symbol=\ensuremath{\wedge},type={symbols},description={The logical conjunction.}}
\newcommand{\symand}{{\glssymbol{symand}}}

\newglossaryentry{symor}{sort={AEsymor},name=\ensuremath{\vee},symbol=\ensuremath{\vee},type={symbols},description={The logical, inclusive disjunction.}}
\newcommand{\symor}{{\glssymbol{symor}}}

\newglossaryentry{symnot}{sort={AFsymnot},name=\ensuremath{\neg},symbol=\ensuremath{\neg},type={symbols},description={The logical negation.}}
\newcommand{\symnot}{{\glssymbol{symnot}}}

%% B group (the group is specified in the sort field)
\newcommand*{\Bgroupname}{Relational Algebra}

% this is the definition of a symbol entry, note that glossaryentry is defined; for convenience using a newcommand, glssymbol is printed
% see also the use of \ensuremath
\newglossaryentry{relation}{sort={BArelation},name=\ensuremath{\mathcal{R}},symbol=\ensuremath{\mathcal{R}},type={symbols},description={A relation. Sometimes with subscript, e.g., \( \relation_{\mathtt{name}} \) to specify name or index.}}
\newcommand{\relation}{\glssymbol{relation}}

\newglossaryentry{rankedrel}{sort={BBrankedrelation},name=\ensuremath{\relation^{\mathcal{O}}},symbol=\ensuremath{\relation^{\mathcal{O}}},type={symbols},description={A ranked, relation that exhibits an ordering of elements induced my \( \mathcal{O} \) . Sometimes used with subscript, e.g., \( \relation_{\mathtt{name}} \) to specify name or index.}}
\newcommand{\rankedrel}{\glssymbol{rankedrel}}

\newglossaryentry{domain}{sort={BCdomain},name=\ensuremath{\mathcal{D}},symbol=\ensuremath{\mathcal{D}},type={symbols},description={A data domain of a relation. Sometimes with subscript, e.g., \(\domain_{\mathtt{name}}\), to specify name or index.}}
\newcommand{\domain}{\glssymbol{domain}}

\newglossaryentry{attribute}{sort={BDattribute},name=\ensuremath{\mathcal{A}},symbol=\ensuremath{\mathcal{A}},type={symbols},description={An attribute of a relation, i.e., a tuple \((\mathtt{name}, \domain)\). Sometimes with subscript to specify name or index.}}
\newcommand{\attribute}{\glssymbol{attribute}}

\newglossaryentry{attributep}{sort={BEattributep},name=\ensuremath{\mathcal{A}^{*}},symbol=\ensuremath{\mathcal{A}^{*}},type={symbols},description={An attribute that acts as a primary key.}}
\newcommand{\attributep}{\glssymbol{attributep}}

\newglossaryentry{attributef}{sort={BFfattributef},name=\ensuremath{\overline{\mathcal{A}}},symbol=\ensuremath{\overline{\mathcal{A}}},type={symbols},description={An attribute that acts as a foreign key.}}
\newcommand{\attributef}{\glssymbol{attributef}}

\newglossaryentry{tuple}{sort={BGtuple},name=\ensuremath{t},symbol=\ensuremath{t},type={symbols},description={A tuple \( \tuple \in \relation \) of attribute values \( a_i \in \domain_i \). Sometimes with subscript to specify the index of the tuple in \(  \relation \).}}
\newcommand{\tuple}{\glssymbol{tuple}}

\newglossaryentry{domainset}{sort={BHdomainset},name=\ensuremath{\mathbb{D}},symbol=\ensuremath{\mathbb{D}},type={symbols},description={The set system of all data domains \( \domain \) supported by a \acrshort{dbms}.}}
\newcommand{\domainset}{\glssymbol{domainset}}

\newglossaryentry{schema}{sort={BIschema},name=\ensuremath{\mathtt{SCH}},symbol=\ensuremath{\mathtt{SCH}},type={symbols},description={The schema of a relation \( \relation \), i.e., all attributes \(\schema (\relation) = (\attribute_1, \attribute_2, \ldots, \attribute_N)\) that make up \(\relation \).}}
\newcommand{\schema}{\glssymbol{schema}}

\newglossaryentry{selection}{sort={BJselection},name=\ensuremath{\sigma},symbol=\ensuremath{\sigma},type={symbols},description={The unary selection operator used to filter a relation \(\relation \). Usually printed with a subscript to describe the boolean predicate, e.g. \( \selection_{\mathcal{P}} \).}}
\newcommand{\selection}{\glssymbol{selection}}

\newglossaryentry{projection}{sort={BKprojection},name=\ensuremath{\pi},symbol=\ensuremath{\pi},type={symbols},description={The unary projection operator used to restrict the attributes \( \attribute \) of a relation \( \relation \). Usually used with subscript to list the desired attributes , e.g. \( \projection_{\attribute_1,\attribute_2} \).}}
\newcommand{\projection}{\glssymbol{projection}}

%% C group (the group is specified in the sort field)
\newcommand*{\Cgroupname}{Multimedia Retrieval}

\newglossaryentry{symdfc}{sort={Csymdfc},name=\ensuremath{\hat{\delta}},symbol=\ensuremath{\hat{\delta}},type={symbols},description={A \gls{dfc} \( \symdfc \colon \domain_q \times \domain_q \times \domain_{1} \cdot \times \domain_{n} \to \symnatural \)}}
\newcommand{\symdfc}{\glssymbol{symdfc}}

\newglossaryentry{distance}{sort={Cdistance},name=\ensuremath{\delta},symbol=\ensuremath{\delta},type={symbols},description={A relation. Sometimes with subscript, e.g., \( \relation_{\mathtt{name}} \) to specify name or index.}}
\newcommand{\distance}{\glssymbol{distance}}


%%%%%%%%%%%%%%%%%%%%%%%%%%%%%%%%%%%%%%%%%%%%%
% Math Symbols.
%%%%%%%%%%%%%%%%%%%%%%%%%%%%%%%%%%%%%%%%%%%%%
\DeclareMathOperator*{\argmin}{argmin}
\DeclareMathOperator*{\argmax}{argmax}

%%%%%%%%%%%%%%%%%%%%%%%%%%%%%%%%%%%%%%%%%%%%%
% Image.
%%%%%%%%%%%%%%%%%%%%%%%%%%%%%%%%%%%%%%%%%%%%%
\newcommand*{\pin}{\includegraphics[height=\heightof{M}]{figures/pin}}