\chapter{On Multimedia Analysis and Retrieval}
\label{chapter:theory_multimedia_analysis_and_retrieval}

%\epiquote{Quote}{Author}
%%%%%%%%%%%%%%%%%%%%%%%%%%%%%%%%%%%%%%%%%%%%%%%%%%%%%%%%%%%%%%%%%%%%%%%%%%%%%%%%%%%%%

\section{Multimedia Data and Multimedia Collections}
\label{section:multmedia_data}
Formalisation of what multimedia data is and what forms it can take (video, audio, images, text + metadata etc.). This formal model has the potential of being an original contribution, since we will make very explicit assumptions about what aspects of a multimedia item there are and which ones are mutable or immutable (e.g, content vs. annotations, metadata, features etc.)

\section{Multimedia Retrieval}

\subsection{Similarity and the Vector Space Model}
\subsection{Approximate Nearest Neighbor Search}
\todo[inline]{Describe techniques for approximate nearest neighbor search (ANN). Focus on a more conceptual overview of the types of algorithms rather than just enumerating concrete examples; this can be used as a build-up for discussing properties of different index structures later. }


\subsection{Beyond Similarity Search}
\todo[inline]{Retrieval and analytics techniques that go beyond simple similarity search (e.g. SOM, summarization, clustering)}

\section{Online Multimedia Analysis}
\todo[inline]{Introducing an online analysis pipeline (e.g., Pythia / Delphi).}

\section{Multimedia Analytics}
\todo[inline]{Describe how the combination of analysis }

\subsection{Beyond Similarity Search}

