\chapter{Introduction}
%\epiquote{Websites that collect quotes are full of mistakes and never check original sources}{Randall Munroe, XKCD}


Multimedia data is ubiquitous and in its various forms it touches almost every aspect of our social and economic everyday life. As the word suggest, multimedia encompasses many different content forms including but not limited to images, videos or audio data as well as any combination thereof. 


\section{Working with Multimedia Data}

\subsection{Multimedia Analysis}

\subsection{Multimedia Analytics}

\subsection{Multimedia Retrieval}

\section{Focus and Significance of Research}

The starting point for this Thesis is the current state-of-the-art in multimedia retrieval and analysis as outlined in the previous sections. Motivated by the ``Ten Research Questions for Scalable Multimedia Analytics''~\cite{Jonson:2016}, it challenges some of the basic assumptions currently employed and operated on and explores the ramifications thereof, with the higher level goal of furthering convergence between research done in \emph{multimedia retrieval, analysis and analytics} on the one hand, and classical \emph{databases} on the other hand.

\begin{description}
    \item[Interactivity] Current state-of-the-art, especially in multimedia retrieval, makes a clear distinction between an \emph{offline} phase during which data is added to a collection, and an online 
\end{description}


