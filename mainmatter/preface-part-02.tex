\chapter*{Preface to Part II}

Since the contributions of this Thesis rely on decades of research in database systems as well as multimedia retrieval and analysis, we will use this part to provide an overview over the relevant fundamentals. Furthermore, we use it to survey related work and to establish the common terminology, which we think is necessary to understand the contributions of this Thesis.

\begin{description}
    \item[\Cref{chapter:theory_databases}]  discusses the theory surrounding \acrlong{dbms} both at a data model and systems level. Most of the fundamentals are guided by the books \emph{DATABASE SYSTEMS The Complete Book} \cite{Garcia:2009Database} and \emph{Database Internals} \cite{Petrov:2019Database}. Furthermore, some information was adapted from a video lecture kindly provided by Andy Pavlo -- Associate Professor at Carnegie Melon University. That lecture is freely available on YouTube\footnote{https://www.youtube.com/c/CMUDatabaseGroup}.
    \item[\Cref{chapter:theory_multimedia_analysis_and_retrieval}] shifts the focus to the analysis and retrieval of multimedia data -- including but not limited to images, audio and video. In addition to the basic techniques, we also discuss concrete system implementations. The two major sources used in this chapter are the books \emph{Multimedia Retrieval} \cite{Blanken:2007multimedia} and \emph{Similarity Search -- The Metric Space Approach} \cite{Zezula:2006Similarity}.
    \item[\Cref{chapter:theory_multimedia_database}] tries to combine the two previous chapters and briefly surveys what efforts have been made to converge the two domains, which are, to this day, still largely disjoint areas of research. This is where the transition to the next part is staged.
\end{description}

All the aforementioned text books and resources provide an excellent overview of their respective fields and this is as a good time as any to thank the authors for their invaluable work. While we do indicate if specific ideas were taken directly from those sources, some information that we considered to be ``general knowledge'', is not always cited explicitly.