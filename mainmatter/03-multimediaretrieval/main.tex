\chapter{On Multimedia Analysis and Retrieval}
\label{chapter:theory_multimedia_analysis_and_retrieval}

%\epiquote{Quote}{Author}
%%%%%%%%%%%%%%%%%%%%%%%%%%%%%%%%%%%%%%%%%%%%%%%%%%%%%%%%%%%%%%%%%%%%%%%%%%%%%%%%%%%%%

\section{Multimedia Data and Multimedia Collections}
\label{section:multmedia_data}
Formalisation of what multimedia data is and what forms it can take (video, audio, images, text + metadata etc.). This formal model has the potential of being an original contribution, since we will make very explicit assumptions about what aspects of a multimedia item there are and which ones are mutable or immutable (e.g, content vs. annotations, metadata, features etc.)

\section{Multimedia Retrieval}

\subsection{Similarity Search and the Vector Space Model}

In multimedia retrieval, there are two important assumptions for similiarity search. These assumptions can be summarized as follows:

\begin{itemize}
    \item For every object $c_{i}$ in a (multimedia) collection $\mathcal{C}$, there exists a feature transformation $\phi \colon \mathcal{C} \to \mathcal{F}$, that maps the object $o_{i} \in \mathcal{C}$ to a feature space $\mathcal{F}$.
    \item The feature space $\mathcal{F}$ and a to be defined distance function $\delta \colon \mathcal{F} \times \mathcal{F} \to \mathbb{R}_{\geq 0}$, constitute a metric space $(\mathcal{F},\delta)$, thus satisfying the non-negativity, identity of indiscernibles, symmetry and subadditivity condition.
\end{itemize}

The output of $\delta$ -- i.e., the calculated distance $d$ -- acts as a proxy for (dis-)similiarty between two objects $o_{i}$, $o_{j} \in \mathcal{C}$ given the feature transformation $\phi$. Hence, the closer two objects $o_{i}$, $o_{j}$ appear under the transformation, the more (dis-)similar they are. For the sake of completeness, it must be pointed out, that whether similarity is directly or inversely proportional to the distance is a matter of definition and depends on the concrete application. Also, in practice, multiple feature transformations $\phi_m$ may exist for a given media collection, leading to different feature spaces $\mathcal{F}_m$ for a collection $\mathcal{C}$ that must be considered jointly. Both aspects are usually addressed by additional \emph{correspondence} and \emph{scoring} functions.

\todo[inline]{Correspondence and score functions, score fusion}

\subsection{Approximate Nearest Neighbor Search}
\todo[inline]{Describe techniques for approximate nearest neighbor search (ANN). Focus on a more conceptual overview of the types of algorithms rather than just enumerating concrete examples; this can be used as a build-up for discussing properties of different index structures later. }


\subsection{Beyond Similarity Search}
\todo[inline]{Retrieval and analytics techniques that go beyond simple similarity search (e.g. SOM, summarization, clustering)}

\section{Online Multimedia Analysis}
\todo[inline]{Introducing an online analysis pipeline (e.g., Pythia / Delphi).}

\section{Multimedia Analytics}
\todo[inline]{Describe how the combination of analysis }

\subsection{Beyond Similarity Search}

