\chapter{On The Design of a Database Management System}
\label{chapter:theory_databases}

\epiquote{Under third normal form, a non-key field must provide a fact about the key, use the whole key, and nothing but the key.}{William Kent}

\gls{dbms}{}, or simply ``databases'', power everything from small and simple websites to large data warehouses that serve millions of users in parallel. Database systems play a crucial role in banking, e-commerce, science, entertainment and practically every aspect of our socio-economic lives. The first commercial \gls{dbms} systems were introduced in the 1960s \cite{Garcia:2009Database} and they have evolved ever since to adapt to a wide range of requirements. Even though many different flavours of \gls{dbms} have emerged over the years, at their core, they still serve the same, fundamental purpose:

\begin{description}
    \item[Management \& Definition] \glsname{dbms} provide mechanisms to structure and organize data corpora that can range from a few megabytes to houndreds of terrabytes in size. For example, data can be structured into documents, relations or graph-like structures.
    \item[Manipulation] \glsname{dbms} provide users with the ability to modify the data using a \gls{dml}. Modifications may include adding, removing or changing existing entries.
    \item[Querying] \glsname{dbms} provide users with the ability to query the data using a \gls{dql}. Such queries can be used to answer specific ``questions'' about the data.
    \item[Guarantees] \glsname{dbms} usually provide guarantees, such as assuring durability upon failure or providing access control for concurrent read and write operations. A well-known set of guarantees is known by its acronym ACID -- which stands for \textbf{A}tomicity, \textbf{C}onsistency, \textbf{I}solation and \textbf{D}urability~\cite{Haerder:1983principles}.
\end{description}

Since the contributions of this Thesis rely on decades of database research, we use this chapter to provide a brief overview over the relevant fundamentals. Most of the basic aspects are inspired and guided by \cite{Garcia:2009Database} and \cite{Petrov:2019Database}.


\section{The Relational Data Model}
\label{section:relational_data_model}

A data model is a formal framework that describes any type of data or information. It usually involves a formal description of the data's \emph{structure}, the \emph{operations} that can be performed and the \emph{constraints} that should be imposed on the data~\cite{Garcia:2009Database}. The purpose of any data model is to formalize how data governed by it can be accessed, modified and queried.

In the context of database systems, it has become common practice to make a clear distinction between the \emph{physical data model} and the a higher-level \emph{logical data model}. While the former captures low-level structure (representation in bits and bytes, data structures etc.) and operations (reads and writes), the latter describes higher-level semantics (attributes, relationships etc.). The argument in favour of this separation is that ``users'' of any \gls{dbms} should not concern themselves with how exactly data is organized and accessed at the lowest level.

In June 1970, E. F. Codd published his pivotal research article \emph{Relational Model of Data for Large Shared Data Banks}~\cite{Codd:1970Relational} in which he describes such a logical data model for databases, which he himself refers to as ``relational''. This model has become the fundament on which many if not most of the modern database management systems have been built in the past decades, with early examples dating back to the 1970s \cite{Astrahan:1976Systemr} and promintent, contemporary examples including systems such as \emph{MySQL}, \emph{PostgreSQL} or \emph{Oracle}. 

The relational model is structured around \emph{relations}, which are a mathematical construct but can be visualized as two-dimensional tables. Such a table consists of columns -- which are called \emph{attributes} -- and rows -- which are called \emph{tuples} and hold \emph{attribute values}. \cref{example:relational_table} features a relation \emph{paintings} and each entry in the table represents a painting and the related attribute values for attributes \emph{title}, \emph{artist} and \emph{painted}.

Semantically, a relation can be seen as a knowledge-base about some fact -- such as, the paintings held by a museum -- under a \gls{cwa}{}\todo{Reference}. That is, the relation contains all the information available about the fact and can thus be used do derive conclusive answers or results given a stated question or query.

\begin{example}[label=example:relational_table]{Table Representation of a Relation $\relation_{paintings}$}{}
    
    The following table lists the content of a ternary relation ($N = 3$) $\relation_{paintings}$. The attributes $\attribute_{title}, \attribute_{artist}, \attribute_{id}$ correspond to the table's columns. The individual tuples $\tuple_i$ are ``valid'' combinations of identifier, painting title and artist and constitute the rows.
        
    \begin{center}
        \begin{tabular}{ l || l | l | l |}
            $\relation_{paintings}$ & $\attributep_{title}$  & $\attribute_{artist}$  & $\attribute_{painted}$ \\ 
            \hline
            \hline
            $t_1$ & Mona Lisa &  Leonardo da Vinci & 1506 \\
            \hline
            $t_2$ & The Starry Night & Vincent van Gogh & 1889 \\
            \hline
            $t_3$ & Las Meninas & Diego Velázquez & 1665 \\
            \hline
        \end{tabular}
    \end{center}
\end{example}

In order to formalize the structure of and the operations that can be executed on the data represented by a relation, one can use \cref{defintion:relation}. 

\begin{definition}[label=defintion:relation]{Relation according to \cite{Codd:1970Relational}}{}
    Given $i,j,N \in \symnatural_{> 0}$ and a collection of sets $\domain_1, \domain_2, ... \domain_N$, a \emph{relation} $\relation$ on these sets is a set of \emph{tuples} $t_i = (a_{i,1}, a_{i,2} ... a_{i,n}) \in \relation$ such that the attribute values $a_{i,1} \in \domain_1, a_{i,2} \in \domain_2 ... a_{i,N} \in \domain_N$. Ergo, $\relation$ is a subset of the Cartesian product of the relation's \emph{data domains} $\domain_j$, that is, $\domain_1 \times \domain_2 ... \times \domain_N$. The number of data domains $N$ is referred to as the relation's \emph{degree} and we call a relation \emph{N-ary}, e.g., binary for $N=2$. 
\end{definition}


The relational model itself does not dictate what the data domains there are. However, in a relational \acrshort{dbms}, the data domains $\domain$ usually correspond to the data types supported by the database, for example, \lstinline{Int} for integer numbers, \lstinline{Float} for floating point numbers or \lstinline{Varchar} for text. We use the set-system $\mathbb{D} = \{ \domain \colon \textrm{if} \: \domain \: \textrm{is supported by DBMS} \}$ to denote the set of all data domains supported by a \gls{dbms}. Furthermore, we use \emph{data domain} synonymous for \emph{data type} \footnote{Strictly speaking, the data domains of a given relation $\relation$ are merely subsets of the sets that represent the respective data type which, in turn, are subsets of even more basic sets such as $\symnatural$ or $\symreal$ for \lstinline{Int} and \lstinline{Float} respectively. The relationship between types and data domains is subject of a more categorical approach to data models and nicely layed out in \cite{Spivak:2009Simplicial}}.

For convenience sake, throughout this thesis, we will use the following notation:
\begin{enumerate*}[label=(\roman*)]
    \item  The combination of a data domain with a human readable label is called \emph{attribute} and we use subscripts to indicate it's name or index, that is, $\attribute_{\mathtt{name}} = (\domain, \mathtt{name})$. 
    \item Attribute values $a_i$ for a tuple $t$ can be addressed by an attribute as follows $a_i = t\left[ \attribute_i \right]$
\end{enumerate*}

In its original form, the relational model assumes the following properties to be true for a relation $\relation$ and its attributes~\cite{Codd:1970Relational}:

\begin{description}
    \item[Ordering of tuples] Tuples $t_i \in \relation$ are inherently unordered and two relations are considered equal if they contain the same tuples, regardless of order.
    \item[Ordering of attributes] Attribute values $a_{i,j}$ always occur in the same order within the $t_i \in \relation$, which corresponds to the order of the attributes $\attribute_j$. This order can evolve over time but remains constant in a momentary snapshot of $\relation$.
    \item[Duplicates] Relations do not allow for duplicates, i.e., every tuple $t_i \in \relation$ must be unique in terms of their attribute values.
    \item[Atomicity] The attribute values in a tuple $a_j \in t_i$ are considered to be atomic, i.e., it is not possible to further decompose them.
\end{description}

Given the notion and the properties of a relation $\relation$, one can introduce the idea of a \emph{heading} or \emph{schema} $\schema(\relation)$ and \emph{body} or \emph{extent} $\extent(\relation)$ of a relation. Which are the set of attributes and tuples respectively that make up $\relation$.

\begin{equation}
    \schema (\relation) = \lbrace \attribute \colon \textrm{if} \: \attribute \: \textrm{is an attribute of} \: \relation \rbrace
\end{equation}

\begin{equation}
    \extent (\relation) = \lbrace t \in \relation \rbrace
\end{equation}

The sum of all data contained in a database system can be regarded as a collection of different relations $\relation, k \in \symnatural_{\geq 0}$ of assorted degrees $N_k$ over data domains $\domain \in \mathbb{D}$. The schema of a database can then be seen as the set system of all $\schema(\relation_k)$. As \cite{Codd:1970Relational} points out, relations are subject to change over time, which are changes to either $\schema(\relation)$ or $\extent(\relation)$, for example, by tuples being added to (insert) or removed from (delete) a relation, or by altering one or multiple attributes.


\subsection{Keys and Normal Forms}

The notion of a relation introduced in the previous section provides us with the basic tools for data modeling. \cite{Codd:1970Relational} proposed a wide range of additional constraints to guarantee proper data definition using the relational model and the following, additional definitions. 

\begin{description}
    \item[Primary key] A \emph{primary key} is a combination of attributes $\mathcal{P} \subset \schema(\relation)$ that uniquely identify a tuple $t_i \in \relation$. That is, the attributes $\schema(\relation) \setminus \mathcal{P}$ and the associated attribute values are functionally determined by $\mathcal{P}$.
    \item[Foreign key] Foreign keys are attributes $\mathcal{F} \subset \schema(\relation)$ that are not primary keys but reference the primary key of $R$ or some other relation $R_k$. Foreign keys can be used to model relationships between relations.
\end{description}

An example of relations with primary and foreign key attributes is given in \cref{example:relational_table_pkfk}. Primary keys are indicated with a * and foreign keys are underlined.

\begin{example}[label=example:relational_table_pkfk]{Relations with Primary and Foreign Keys}{}
    
    The following tables lists the schema and extent of $\relation_{\mathtt{paintings}}$ and $\relation_{\mathtt{artists}}$.
    \begin{center}
        \begin{tabular}{ l || l | l | l |}
            $\relation_{\mathtt{painting}}$ & $\attributep_{\mathtt{title}}$  & $\attribute_{\mathtt{artist}}$  & $\attribute_{\mathtt{painted}}$ \\ 
            \hline
            \hline
            $t_1$ & Mona Lisa &  Leonardo da Vinci & 1506 \\
            \hline
            $t_2$ & The Starry Night & Vincent van Gogh & 1889 \\
            \hline
            $t_3$ & Las Meninas & Diego Velázquez & 1665 \\
            \hline
        \end{tabular}
    \end{center}

    \begin{center}
        \begin{tabular}{ l || l | l | l | l |}
            $\relation_{\mathtt{artist}}$ & $\attributep_{\mathtt{fname}}$ & $\attributep_{\mathtt{lname}}$ & $\attribute_{\mathtt{birth}}$ & $\attribute_{\mathtt{death}}$\\ 
            \hline
            \hline
            $t_1$ & Leonardo & da Vinci & 1452 & 1519 \\
            \hline
            $t_2$ & Vincent & van Gogh & 1853 & 1890 \\
            \hline
            $t_3$ & Diego & Velázquez & 1599 & 1660 \\
            \hline
        \end{tabular}
    \end{center}

    Note, that we assume here that people (artists) are uniquely identified by their first and lastname, which is obviously a simplification.
\end{example}

Given the notion of primary and foreign keys, Codd proposed a series of \emph{normal forms} that are an indication of the normalization and transitively the quality of a data model.

\begin{description}
    \item[\acrfull{1nf}] requires, that no attribute of in a relation has other relations as values (atomicity).
    \item[\acrfull{2nf}] requires, that every attribute that is not part of the primary key is functionally determined by the whole primary key and not any subset thereof. Given $\relation_{\mathtt{artist}}$ in \Cref{example:relational_table_pkfk}, this means that all of the non-prime attributes must be determined by $\attributep_{\mathtt{fname}}$ and $\attributep_{\mathtt{lname}}$ jointly.
    \item[\acrfull{3nf}] requires, that every attribute that is not part of the primary key is functionally determined solely by the primary key and that they do not depend on any other key. Given $\relation_{\mathtt{artist}}$ in \Cref{example:relational_table_pkfk}, this means that all of the non-prime attributes must be determined by $\attributep_{\mathtt{fname}}, \attributep_{\mathtt{lname}}$ alone\footnote{The normal forms introduced thus far have also led to the mnemonic \emph{``The key, the whole key, and nothing but the key, So help me Codd.''} in reference to a similar sentence used as a vow in courts of law.}.
\end{description}

All the normal forms build onto one another, i.e., for a data model to be considered \acrshort{3nf} is also required to satisfy \acrshort{2nf}  and \acrshort{1nf}. Additional normal forms up to \emph{6NF} and the \acrfull{bcnf} have been defined. For the sake of brevity, we will ommit those since they are not relevant for the further discussion.

\subsection{Relational Algebra}
\label{section:rel_algebra}

Having introduced the aspect of data representation, we can now move to that of operations that can be performed on the data. For the sake of brevity, we will limit ourselves to query operations. For expressing queries, \cite{Codd:1970Relational} proposed the idea of a \emph{relational algebra}, which follows a simple yet powerfule idea: All query operations performed on relations are expressed by a set of \emph{relational operations} that take one or multiple relations as input and output a new relation as expressed by \Cref{equation:rel_op}.

\begin{equation}
    \label{equation:rel_op}
    \mathtt{OP}: \relation_1, \ldots, \relation_n \rightarrow \relation_{O}
\end{equation}

Those relational operators can then be composed to express a query of arbitrary complexity as indicated by \Cref{equation:rel_query}.

\begin{equation}
    \label{equation:rel_query}
    \mathtt{QUERY} = \mathtt{OP_{1}} \circ \mathtt{OP_{2}}, \ldots , \circ \mathtt{OP_{m}}
\end{equation}

In addition to this idea, Codd proposed the minimal set of relational operators listed in \Cref{table:relational_operators} and explained in the following sections. For the sake of completeness, it must be mentioned that notation and operators may slightly differ depending on the literature. We mainly use \cite{Garcia:2009Database} as our reference.

\begin{table}
    \caption{The relational operators proposed by Codd et al \cite{Codd:1970Relational,Garcia:2009Database}.}
    \label{table:relational_operators}
    \begin{tabular}{| l | c | c | p{75mm} |}
        \hline
       \textbf{Name} & \textbf{Symbol} & \textbf{Arity}  & \textbf{Example} \\ 
        \hline
        \hline
        Union & $\cup$  & 2 & Set union of two input relations. \\
        \hline
        Intersection & $\cap$  & 2 & Set intersection of two input relations. \\
        \hline
        Difference & $\setminus$  & 2 & Set difference of two input relations. \\
        \hline
        Cartesian Product & $\times$ & 2 & Pairs each tuple from one the left with every tuple of the right input relation and concatenates them. \\
        \hline
        Selection & $\selection_{\mathcal{P}}$ &  1 & Removes tuples from the input relation that don't match predicate $\mathcal{P}$. \\
        \hline
        Projection & $\projection_{\chi}$ &  1 & Removes attributes from the input relation that are not included in $\chi$. \\
        \hline
        Natural join & $\Join_{\mathcal{P}}$ & 2 & Pairs each tuple from the left with every tuple of the right input relation if their shared attributes match and concatenates them. \\
        \hline
    \end{tabular}
\end{table}

\subsubsection{Simple Set Operations $\cup$,$\cap$ and $\setminus$}

Since relations are sets of tuples, all basic operations known from set theory can be applied with the only constraint that the two input relations $\relation_L,\relation_R$ must be \emph{union compatible} and thus exhibit the same attributes, i.e. $\schema(\relation_L) = \schema(\relation_R)$.

The set union $\relation_L \cup \relation_R$ generates a new relation of all tuples contained in either $\relation_L$ OR $\relation_R$, as expressed by \Cref{equation:rel_op_union}

\begin{equation}
    \label{equation:rel_op_union}
    \relation_L \cup \relation_R = \{ t \colon t \in \relation_L \symor t \in \relation_R \}
\end{equation}

The intersection $\relation_L \cap \relation_R$ generates a new relation of all tuples contained in $\relation_L$ AND $\relation_R$, as expressed by \Cref{equation:rel_op_intersection}.

\begin{equation}
    \label{equation:rel_op_intersection}
    \relation_L \cup \relation_R = \{ t \colon t \in \relation_L \symand t \in \relation_R \}
\end{equation}

The difference $\relation_L \setminus \relation_R$ generates a new relation of all tuples contained in $\relation_L$ AND NOT in $\relation_R$, as expressed by \Cref{equation:rel_op_difference}.

\begin{equation}
    \label{equation:rel_op_difference}
    \relation_L \cup \relation_R = \{ t \colon t \in \relation_L \symand t \notin \relation_R \}
\end{equation}

Set operations simply combine two input relations using set semantics whithout changing the structure of the relation, i.e., $\schema(R_{I}) = \schema(R_{O})$. Due to relations being sets, duplicates resulting from the application of a union operation are \emph{implictly} eliminated.

\subsubsection{Cartesian Product}
The binary, \emph{cartesian product} or \emph{cross product} $\relation_L \times \relation_R$ of two input relations $\relation_L,\relation_R$ concatenates every tuple $t_{L} \in \relation_L$ with every tuple $t_{R} \in \relation_R$ to form a new output tuple, as expressed by \Cref{equation:rel_op_cartesian}.

\begin{equation}
    \label{equation:rel_op_cartesian}
    \relation_L \times \relation_R = \{ (t_L, t_R) : t_L \in \relation_L \wedge t_R \in \relation_R \}
\end{equation}

The result is a relation that contains all the attributes of $\relation_L$ and $\relation_R$, i.e., $\schema(\relation_L \times \relation_R) = \schema(\relation_L) \cup \schema(\relation_R)$ and every possible permutation of tuples from the input relations.


\subsubsection{Selection}

The unary, generalized \emph{selection} operator $\selection_{\phi}(\relation)$ applied on an input relation $\relation$ creates an output relation that contains a subset of tuples $t \in \relation$ such that only tuples that match the predicate $\phi$ are retained as expressed by \Cref{equation:rel_op_selection}.

\begin{equation}
    \label{equation:rel_op_selection}
    \selection_{\phi}(\relation) = \{ t \in \relation \colon \phi(t) \} \subset \relation
\end{equation}

The predicate $\phi$ can be any conditional statement consisting of individual atoms involving attributes of $\relation$ or any constant value combined by logical operators $\symand$, $\symor$ or $\symnot$. Examples could be $\attribute_1 \geq 2$ (to express that an attribute should be greater to a constant) or $\attribute_2 = \attribute_3$ (to express that an attribut must be equal to another attribute) or $\attribute_1 \geq 2 \wedge \attribute_2 = \attribute_3$ to combine the two with $A_1,A_2,A_3 \in \schema(\relation)$.

\subsubsection{Projection}
The unary \emph{projection} operator $\projection_{\chi}(\relation)$ with $\chi \subset \schema(\relation)$ applied on an input relation $\relation$ creates an output relation that only contains the attributes listed in $\chi$ as expressed by \cref{equation:rel_op_projection}

\begin{equation}
    \label{equation:rel_op_projection}
    \projection_{\chi}(\relation) = \{ t\left[ \chi \right] : t \in \relation \} \: \textrm{with} \: t\left[ \chi \right] = \{ t[\attribute] \colon \attribute \in \chi \}
\end{equation}

Conseqeuently, $\schema(\projection_{\chi}(\relation)) = \chi$. All the tuples in $\relation$ are retained in the process, however, resulting duplicates are implicity removed.

\subsubsection{Natural Join}

The binary, \emph{natural join} operator $\relation_L \Join \relation_R$ on two input relations $\relation_L$, $\relation_R$ concatenates every tuple $t_{L} \in \relation_L$ with every tuple $t_{R} \in \relation_R$ to form a new output tuple, if the attribute values of $t_{L}$ and $t_{R}$ are the same for the shared attribues $\xi = \{ \attribute : \attribute \in \schema(\relation_L) \symand \attribute \in \schema(\relation_R) \}$ in $\relation_L$ and $\relation_R$ as expressed by \Cref{equation:rel_op_join}

\begin{equation}
    \label{equation:rel_op_join}
    \relation_L \Join \relation_R = \{ (t_L, t_R) : t_L \in \relation_L \symand t_R \in \relation_R \symand t_L\left[ \xi \right] = t_R\left[ \xi \right] \}
\end{equation}

The result is a relation that contains all the attributes of $\relation_L$ and $\relation_R$, i.e., $\schema(\relation_L \times \relation_R) = \schema(\relation_L) \cup \schema(\relation_R)$. If two relations coincide in all their attributes, the natural join becomes a cartesian product. Furthermore, the natural join can be expressed as a concatenation of the cartesian product with a selection. For example, given a relation $\relation_L$ with $\schema(\relation_L) = (\attribute_A, \attribute_B, \attribute_C)$ and $R_R$ with $\schema(\relation_R) = (\attribute_B, \attribute_D, \attribute_F)$, then $\relation_L \Join \relation_R \equiv \selection_{\attribute_{L.B} = \attribute_{R.B}}(\relation_L \times \relation_R)$.

\subsubsection{Expressing Queries}

The following \Cref{example:rel_alg_query} illustrates how relational operators can be combined to form complex queries using a simple toy example.

\begin{example}[label=example:rel_alg_query]{Query on $\relation_{\mathtt{paintings}}$ Expressed in Relational Algebra}{}
    
    The following table lists the content of the ternary relation $\relation_{\mathtt{paintings}}$. 
        
    \begin{center}
        \begin{tabular}{ l || l | l | l |}
            $\relation_{\mathtt{paintings}}$ & $\attributep_{\mathtt{title}}$  & $\attribute_{\mathtt{artist}}$  & $\attribute_{\mathtt{painted}}$ \\ 
            \hline
            \hline
            $t_1$ & Mona Lisa &  Leonardo da Vinci & 1506 \\
            \hline
            $t_2$ & The Starry Night & Vincent van Gogh & 1889 \\
            \hline
            $t_3$ & Las Meninas & Diego Velázquez & 1665 \\
            \hline
        \end{tabular}
    \end{center}

    The query ``return the names of all paintings that have been painted after 1600'' can be expressed by a selection on $\relation_{\mathtt{paintings}}$ followed by projection as follows: $\projection_{A_{\mathtt{title}}} ( \selection_{A_{\mathtt{painted} > 1600}} ( \relation_{\mathtt{paintings}}))$ which results in: 

    \begin{center}
        \begin{tabular}{ l || l |}
            $\relation_{\mathtt{result}}$ & $\attributep_{\mathtt{title}}$  \\ 
            \hline
            \hline
            $t_3$ & Las Meninas \\
            \hline
        \end{tabular}
    \end{center}
\end{example}

\subsection{Extensions}
\label{section:rel_extensions}

While the relational model and the relational algebra introduced in the previous section forms the foundation of many modern \glsname{dbms}, the model as originally proposed by Codd is too limited to accomodate the functionality described by the SQL standard. For example, many databases actually need to store duplicate data and therefore, the notion of a relation -- which is a set of tuples, i.e., does not allow for duplicates -- is inadequate. 

\subsubsection{Relations vs. Bag vs. Sequences}

\subsubsection{Extended Projection}

\subsubsection{Ranked Relational Algebra}

\subsubsection{Similarity Search and the Relational Model}

Using the relationship between (diss-)similarity and distance, it has been shown by Giangreco et al., that for a database system to be able to support similarity search given the relational model for databases as described in \cref{section:relational_data_model}, one can extend the set system of allowed data domains $\mathbb{D}$ by $\domain \subset \symreal^{dim}, dim \in \symnatural_{>1}$ and postulate the existence of a relational similarity operator $\tau_{\delta(\cdot,\cdot),a,q}(R)$ that \emph{``performs a similarity query under a distance $\delta(\cdot,\cdot)$ applied on an attribute $a$ of relation $R$ and compared to a query vector $q$.''} (\cite{Giangreco:2018thesis}, p. 138). Such an operation introduces an implicit attribute in the underlying relation $\relation$, which in turn induces an ascending ordering of the tuples. Using this operation, one can go on to define two concrete implementations, namely $\tau^{kNN}_{\delta(\cdot,\cdot),a,q}(\relation)$, and $\tau^{\epsilon NN}_{\delta(\cdot,\cdot),a,q}(\relation)$, which limit the number of retrieved results by their cardinality $k$ or a maximum cut-off distance $\epsilon$ respectively.

While postulating a new, relational operation $\tau^{kNN}_{\delta(\cdot,\cdot),a,q}(\relation)$ or $\tau^{\epsilon NN}_{\delta(\cdot,\cdot),a,q}(\relation)$ as proposed by \cite{Giangreco:2018thesis} has a certain elegance to it, it comes with limitations that become apparent upon disection of its structure. In its postulated form, $\tau$ addresses several functions at once:

\begin{enumerate}
    \item It specifies the distance function that should be evaluated and the provided arguments.
    \item It generates an implicit distance attribute on the underlying relation $\relation$.
    \item It imposes an ascending ordering and limits the cardinality of $\relation$ based on a predicate or a specified limit.
\end{enumerate}

While being very specific and thus straightforward to implement, the amalgamation of all this functionality into a single operation is very specifically tailored to the use-case of similarity search and only of limited value when considering more general proximity-based query operations. If one would, for example, want to obtain the $k$ farthest neighbours rather than the $k$ nearest neighbours, as necessary when doing MIPS or obtaining negative examples, we would have to either change the distance function or extend the definition of $\tau$. 

Another important issue with the definition of $\tau$ in its current form is that despite the two opreations $\tau^{kNN}_{\delta(\cdot,\cdot),a,q}(\relation)$ and $\tau^{\epsilon NN}_{\delta(\cdot,\cdot),a,q}(\relation)$ serving a very similar purpose, they behave very differently with respect to other operations. Generally, $\tau^{kNN}_{\delta(\cdot,\cdot),a,q}(\relation)$ does not commute with any selection $\sigma$ due to the inherent limiting of the cardinality to a constant value, hence:

\begin{equation}
    \label{equation:commutation_of_tau}
    \sigma(\tau^{kNN}_{\delta(\cdot,\cdot),a,q}(\relation)) \neq \tau^{kNN}_{\delta(\cdot,\cdot),a,q}(\sigma(\relation))
\end{equation}

The left-hand side of \cref{equation:commutation_of_tau} filters the results of a kNN-search on $\relation$, thus returning $n \leq k$ results, wherein $n = k$ only if $\sigma$ matches all tuples. The right-hand side of \cref{equation:commutation_of_tau} performs a kNN-search on a pre-filtered relation $\relation$, also returning $n \leq k$ entries. However, $n$ will only be smaller than $k$ if $\sigma$ selects fewer than $k$ tuples.

The same isn't true for $\tau^{\epsilon NN}_{\delta(\cdot,\cdot),a,q}(\relation)$, due to the limitation of the cardinality being facilitated by an \emph{implicit} selection $\sigma_{\delta \leq \epsilon}$.

\begin{equation}
    \sigma(\tau^{\epsilon NN}_{\delta(\cdot,\cdot),a,q}(\relation)) = \tau^{\epsilon NN}_{\delta(\cdot,\cdot),a,q}(\sigma(\relation))
\end{equation}

Hence, $\sigma$ and $\tau^{\epsilon NN}_{\delta(\cdot,\cdot),a,q}(\relation)$ commute and yield equivalent results as long as $\sigma$ does not rely on the distance attribute introduced by $\tau^{\epsilon NN}_{\delta(\cdot,\cdot),a,q}(\sigma(\relation))$

\section{Queries: From Expression to Execution}

\subsection{Structured Query Language (SQL)}

\subsection{Query Planning}

\subsection{Query Execution}

\section{Storage, Indexes and Caching}

\section{Architectual Considerations}