%%%%%%%%%%%%%%%%%%%%%%%%%%%%%%%%%%%%%%%%%%%%%%%%%%%%%%%%%%%%%%%%%%%%%%%%%%%%%
%%  ZUSAMMENFASSUNG
%%%%%%%%%%%%%%%%%%%%%%%%%%%%%%%%%%%%%%%%%%%%%%%%%%%%%%%%%%%%%%%%%%%%%%%%%%%%%

\chapter{Zusammenfassung}

\begin{otherlanguage}{ngerman}
Multimediale Daten in ihren verschiedensten Ausprägungen bergen eine einmalige Herausforderung bei deren Speicherung und Verwaltung, speziell wenn die Durchsuchbarkeit und Analyse von grossen Beständen ermöglicht werden soll. Die inheränte Unstrukturiertheit der Daten selbst und der ``Fluch der Dimensionalität'', welcher den Umgang mit den daraus abgeleiteten Repräsentationen erschwert, bringen eine Vielzahl von Problemen mit sich, welchen auf unterschiedlichsten Ebenen begegnet werden kann. Dies ist Ausganglage für ein Forschungsgebiet, welches sich mit der effizienten und effektiven Suche in multimedialen Datensammlungen beschäftigt und aus dem über die Jahre eine Vielzahl von spezialisierten Systemen und Anwendungen enstanden sind.

Fortschritte im Bereich der interaktiven, multimedialen Suche sowie der multimedialen Analytik haben allerdings gezeigt, dass eine Vielzahl der Annahmen, welche für die klassische Multimedia-Suche getroffen werden, für solche interaktive Anwendungsgebiete nicht oder nur beschränkt gelten. Einerseits können eine Vielzahl der Anforderungen nicht durch einfache Ähnlichkeitssuche ausgedrückt werden und benötigen deshalb ein Abfragemodel, welches die notwendige Flexibilität bringt. Andererseits ist die oft getroffene Annahme, dass Datenbestände sich zur Abfragezeit nicht verändern, in der Praxis nicht haltbar, insbesondere da Analytik das explizite Ziel verfolgt, neue Informationen und Wissen zu generieren und zu speichern. In Konsequenz ist es angesichts der vielen möglichen Anfragearten unvernünftig anzunehmen, dass ein/e Benutzer/in dem System vorgeben kann, wie diese jeweils zu beantworten sind.
 
Ausgehend von diesen Lücken und motiviert von der Tatsache, dass sehr ähnliche Herausforderungen bereits im Kontext klassicher Datenbankanwendungen für strukturierte Daten angegangen und gelöst wurden, präsentiert diese Disseration drei Beträge um den zuvor erwähnten Problemen zu begegnen. Wir präsentieren eine Abfragemodel, welches die Idee von distanzbasierter Suche verallgemeinert und formalisieren die Verbindung solcher Abfragen zu hoch-dimensionalen Indexstrukturen. Dies wird komplementiert durch ein Kostenmodel, welches den oft impliziten Kompromiss zwischen Ausführungsgeschwindigkeit und Qualität  der Resultate bei der Verwendung dieser Indexstrukturen transparent macht. Zu guter letzt beschreiben wir ein Model, welches die persistente und transaktionale Anlage und Pflege solche Indexstrukturen ermöglicht.

Alle erwähnten Beiträge sind in einem öffentlich zugänglichen und frei lizenzierten Datenbanksystem \cottontail{} implementiert, auf dessen Basis wir eine Evaluation präsentieren, welche die Effektivität der Modele demonstriert. Wir schliessen mit einer Diskussion über mögliche Richtungen für zukünftige Entwicklungen mit Blick auf eine weitere Konvergenz zwischen der Forschung in Datenbank- und (interaktiven) Suchsystemen für die multimediale Datenbestände.
\end{otherlanguage}

\cleardoublepage
